\documentclass[10pt]{article}

\usepackage[utf8]{inputenc}

\usepackage{amsmath,amssymb}

\usepackage{esvect}

\usepackage{listings}

\usepackage{color}

\usepackage{graphicx}

\usepackage{float}

\usepackage{blindtext}

\usepackage{tabularx}

\PassOptionsToPackage{hyphens}{url}\usepackage{hyperref}
\hypersetup{
	colorlinks=true,
	linkcolor=blue,
	filecolor=magenta,      
	urlcolor=cyan,
}


\title{Middleware technologies for distributed systems\\Project 1}

\date{2020-2021}



\begin{document}
	\begin{titlepage}
		\begin{figure}[t]
			\centering\includegraphics[width=0.7\textwidth]{../../docResources/logo_polimi}
		\end{figure}
		\maketitle
		
		\large
		\begin{tabularx}{\linewidth}{@{}lXl@{}}
			\textit{Authors:}  & & \textit{Professors:} \\
			Andrea Falanti      & & Prof.\@ Luca Mottola\\
			Federico Ferri  & & Prof. Alessandro Margara\\
			Abanoub Faltaous & & \\
		\end{tabularx}		
		\thispagestyle{empty}
	\end{titlepage}
	
	\tableofcontents
	\newpage
	
	\section{Introduction}
	The project is about creating a system for managing contact tracing among multiple IoT devices. Contacts happens when two devices are within the same broadcast domain.The contacts among devices are reported periodically to the backend on the regular Internet. Whenever a device signals an event of interest, all the devices that had a contact with it must be notified. The project is based on the fact that all nodes are constantly reachable from a static IoT device that acts like an IPv6 border router.   
	
	
	\section{Implementation}
	\subsection{Technologies used}
	The implementation requires to use the COOJA simulator of ContikiNG for the network simulation of the IoT devices. For the backend we decide to use Akka. We decide to use publish/subscribe architecture style for the communication between the simulator and Akka. The communication happens through MQTT both from Akka either COOJA. We decide also to use the broker Mosquitto, that are (in our case) already on the machine, to finalize the publish/subscribe through MQTT.
	
	\subsection{ContikiNG}
	For the simulation of the IoT devices we use nodes to publish and subscribe and a rpl border router to allow nodes to connect to the broker on our machine. Nodes implement three main functionalities:
	one is to send in broadcast to all other visible nodes its own id, this is implemented:
	
	\begin{itemize}
		\item 	First,  nodes use a state machine to connect with mqtt to the message broker and then publish data periodically(these data represent the events of interest)      
		\item   Second,	 is to send in broadcast to all other visible nodes its own id, this is implemented using udp on an IP address set to the link local all-nodes multi-cast address. When a node receives this kind of message, it calls a callback, that allows that node to publish its own id and the id of the sender of the message if it can
		\item	Third, nodes receive data from their subscription through MQTT	
	\end{itemize}
	
	In our implementation there are three topics: one when two people are in contact, one for the event of interest and one for the notification. Our nodes publish on the first two topic, while are subscribed to the third one, vice versa for Akka that is subscribed to the first two topics while for the third one publish. When a node from the simulation send a publish on a topic the broker on our machine takes care to publish it on a broker server, we decide to publish on test.mosquitto.org:1883. 
	We need a bridge to publish/subscribe from local broker to remote broker, so we need to configure the local broker as follow:\\
	\\
	\texttt{connection bridge-01 \\
	address test.mosquitto.org:1883 \\
	topic \# out 1 \\
	topic \# in 1  \\}
	\\
	The first line is the connection name that we choose ('bridge-01'). The second line says the address of the remote broker. The last two rows allow to bridge all topic since we are using the wildcard '\#', 'in' specifies on which topic the broker can recieve from the remote broker and 'out' specifies on which topic the broker can publish on the remote broker, in the end '1' express the QoS (quality of service). Moreover to connect correctly to our broker we need define a unique client-id, username and a password (username and password not need to be unique). 
	
	\subsection{Akka}
	The backend server is realized by using the Akka middleware and its composed by three main parts:
	
	\begin{itemize}
		\item the server main file, that bootstrap the server and the actor hierarchy. At launch, it initialize the server actor, then it creates the MQTT connection with the broker, subscribes to the topics where the IoT devices publish the messages and provides the functions to deserialize their json content to build actual message instances for the server actors.
		\item the server actor, that has the references to all device actors, which are instanciated as children of it. It receives all the messages from the server main and distribute them to the correct device actor. If the IoT device that have sent the message is not present in the actor hierarchy, the respective device actor is created as a child of the server actor. When it receives notification messages from the device actors, it sends a message on the apposite topic of MQTT, so that IoT devices can get the notification. Server actor also perform a simple fault tolerance strategy, restarting the device actors in case of any exception, keeping also their previous state so that contact list is not lost.
		\item the device actors, that represent a digital abstraction of the physical IoT devices. In our case they simply store the set of ids of devices with which they have contacts. When they receive contact messages they simply add the ids to the set. When they receive an event message, they produce and send notification messages to the server actor, one for each device to contact.
	\end{itemize}      
	
	
	
\end{document}